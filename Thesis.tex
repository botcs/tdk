%% ----------------------------------------------------------------
%% Thesis.tex -- MAIN FILE (the one that you compile with LaTeX)
%% ---------------------------------------------------------------- 

% Set up the document
\documentclass[a4paper, 11pt, oneside]{Thesis}  % Use the "Thesis" style, based on the ECS Thesis style by Steve Gunn
\graphicspath{Figures/}  % Location of the graphics files (set up for graphics to be in PDF format)

%For accentuated characters
\usepackage[utf8]{inputenc}
\usepackage[T1]{fontenc}

% Include any extra LaTeX packages required
\usepackage[square, numbers, comma, sort&compress]{natbib}  % Use the "Natbib" style for the references in the Bibliography
\usepackage{verbatim}  % Needed for the "comment" environment to make LaTeX comments

\usepackage{vector}  % Allows "\bvec{}" and "\buvec{}" for "blackboard" style bold vectors in maths
\hypersetup{urlcolor=blue, colorlinks=true}  % Colours hyperlinks in blue, but this can be distracting if there are many links.

%% ----------------------------------------------------------------
\begin{document}
\frontmatter      % Begin Roman style (i, ii, iii, iv...) page numbering

% Set up the Title Page
\title  {Building, testing and visualizing neural networks from scratch}
\authors  {\texorpdfstring
            {\href{https://www.github.com/botcs}{Csaba Botos}}
            {Csaba Botos}
            }
\addresses  {\groupname\\\deptname\\\univname}  % Do not change this here, instead these must be set in the "Thesis.cls" file, please look through it instead
\date       {\today}
\subject    {}
\keywords   {}

\maketitle
%% ----------------------------------------------------------------

\setstretch{1.3}  % It is better to have smaller font and larger line spacing than the other way round

% Define the page headers using the FancyHdr package and set up for one-sided printing
\fancyhead{}  % Clears all page headers and footers
\rhead{\thepage}  % Sets the right side header to show the page number
\lhead{}  % Clears the left side page header

\pagestyle{fancy}  % Finally, use the "fancy" page style to implement the FancyHdr headers

%% ----------------------------------------------------------------
% Declaration Page required for the Thesis, your institution may give you a different text to place here
\Declaration{

\addtocontents{toc}{\vspace{1em}}  % Add a gap in the Contents, for aesthetics

I, Botos Csaba, declare that this thesis titled 'Building, testing and visualizing neural networks from scratch', and the work presented in it are my own. I confirm that:

\begin{itemize} 
\item[\tiny{$\blacksquare$}] This work was done wholly or mainly while in candidature for a research degree at this University.
 
\item[\tiny{$\blacksquare$}] Where any part of this thesis has previously been submitted for a degree or any other qualification at this University or any other institution, this has been clearly stated.
 
\item[\tiny{$\blacksquare$}] Where I have consulted the published work of others, this is always clearly attributed.
 
\item[\tiny{$\blacksquare$}] Where I have quoted from the work of others, the source is always given. With the exception of such quotations, this thesis is entirely my own work.
 
\item[\tiny{$\blacksquare$}] I have acknowledged all main sources of help.
 
\item[\tiny{$\blacksquare$}] Where the thesis is based on work done by myself jointly with others, I have made clear exactly what was done by others and what I have contributed myself.
\\
\end{itemize}
 
 
Signed:\\
\rule[1em]{25em}{0.5pt}  % This prints a line for the signature
 
Date:\\
\rule[1em]{25em}{0.5pt}  % This prints a line to write the date
}
\clearpage  % Declaration ended, now start a new page

%% ----------------------------------------------------------------
% The "Funny Quote Page"
\pagestyle{empty}  % No headers or footers for the following pages

\null\vfill
% Now comes the "Funny Quote", written in italics
\textit{``Write a funny quote here.''}

\begin{flushright}
If the quote is taken from someone, their name goes here
\end{flushright}

\vfill\vfill\vfill\vfill\vfill\vfill\null
\clearpage  % Funny Quote page ended, start a new page
%% ----------------------------------------------------------------

% The Abstract Page
\addtotoc{Abstract}  % Add the "Abstract" page entry to the Contents
\abstract{
\addtocontents{toc}{\vspace{1em}}  % Add a gap in the Contents, for aesthetics

Recently a special branch of Machine Learning, a model based on living organic systems called Deep Neural Networks overtakes previous paradigm of algorithmic problem solving. 
It gained larger attention when better results were achieved than task-specific, handcrafted models in feature extraction. 
%Despite the computational cost of the training process, the generalization capabilities and the scalability of recent architectures make Deep Learning a powerful tool applied widespread in computer vision, speech recognition, language processing. 
%Today's leading companies (Google, Facebook, Ebay, Twitter etc\dots) are the main investors and beneficiaries of researches and developement of DNNs. 
The reason behind its success is their scalability: the latest architectures are able to exploit the capacity of cutting-edge GPU hardwares since the abstraction of the data are accomplished by succeeding neural nodes performing elementary operations which can be easily paralellized.

It is of the utmost importance to understand the main concept of such networks to contribute to the breakthroughs of the fourth industrial revolution. 
To this purpose, building a framework from the base unit blocks of the newest models is the best introduction to Machine Learning. 
In my research I have disassembled black-box representated networks to the very basic, intuitive level and reorganized it in O.O. manner where each neural layer is treated as an entity derived from a common ancestor, therefore information flow and processes of the system are easily traced. 
Current implementations are based on the principals of the components used by networks built for ImageNet classification, such as Convolutional, ReLU, Max-Pooling, Fully-Connected, Softmax and k-Winner-Takes-All layers.
Furthermore the following training methods and policies were adapted: cross validated, minibatch, on-line, $L_p$ regularized and basically Stochastic Gradient Descent training.

For testing the framework, the parameter space of Fully Connected networks was exhaustively explored. 
After train and evaluating sessions - mainly performed on the MNIST and self-acquisited datasets - the results were gathered to analyze the performance of different architectures. 
For further investigation the best performing models  were compared to each other to find pros and cons of different capacity, layout and training of networks.

Besides architectural experiments, targeted by many recent researches a non-trivial task of visualizing the inner representation of information, understanding transient activation patterns was studied as well.
Previously mentioned candidate networks were also visualized individually to retrieve information about characteristics of the processes in their Hidden Layer.
%Novel visualization methods of Ian J. Goodfellow et al., Matthew Zeiler, Ross Girshick et al., summarized by Andrej Karpathy, were analyzed and compared.
My implementation proposes a simplification of the DeConvNet derived from Gradient Ascent, efficient algorithm to reveal patterns recognized by nodes in the hidden layers of Neural Networks, to produce adversarial input samples.



}

\clearpage  % Abstract ended, start a new page
%% ----------------------------------------------------------------

\setstretch{1.3}  % Reset the line-spacing to 1.3 for body text (if it has changed)

% The Acknowledgements page, for thanking everyone
\acknowledgements{
\addtocontents{toc}{\vspace{1em}}  % Add a gap in the Contents, for aesthetics

The acknowledgements and the people to thank go here, don't forget to include your project advisor\ldots

}
\clearpage  % End of the Acknowledgements
%% ----------------------------------------------------------------

\pagestyle{fancy}  %The page style headers have been "empty" all this time, now use the "fancy" headers as defined before to bring them back


%% ----------------------------------------------------------------
\lhead{\emph{Contents}}  % Set the left side page header to "Contents"
\tableofcontents  % Write out the Table of Contents

%% ----------------------------------------------------------------
\lhead{\emph{List of Figures}}  % Set the left side page header to "List if Figures"
\listoffigures  % Write out the List of Figures

%% ----------------------------------------------------------------
\lhead{\emph{List of Tables}}  % Set the left side page header to "List of Tables"
\listoftables  % Write out the List of Tables

%% ----------------------------------------------------------------
\setstretch{1.5}  % Set the line spacing to 1.5, this makes the following tables easier to read
\clearpage  % Start a new page
\lhead{\emph{Abbreviations}}  % Set the left side page header to "Abbreviations"
\listofsymbols{ll}  % Include a list of Abbreviations (a table of two columns)
{
% \textbf{Acronym} & \textbf{W}hat (it) \textbf{S}tands \textbf{F}or \\
\textbf{LAH} & \textbf{L}ist \textbf{A}bbreviations \textbf{H}ere \\

}

%% ----------------------------------------------------------------
\clearpage  % Start a new page
\lhead{\emph{Physical Constants}}  % Set the left side page header to "Physical Constants"
\listofconstants{lrcl}  % Include a list of Physical Constants (a four column table)
{
% Constant Name & Symbol & = & Constant Value (with units) \\
Speed of Light & $c$ & $=$ & $2.997\ 924\ 58\times10^{8}\ \mbox{ms}^{-\mbox{s}}$ (exact)\\

}

%% ----------------------------------------------------------------
\clearpage  %Start a new page
\lhead{\emph{Symbols}}  % Set the left side page header to "Symbols"
\listofnomenclature{lll}  % Include a list of Symbols (a three column table)
{
% symbol & name & unit \\
$a$ & distance & m \\
$P$ & power & W (Js$^{-1}$) \\
& & \\ % Gap to separate the Roman symbols from the Greek
$\omega$ & angular frequency & rads$^{-1}$ \\
}
%% ----------------------------------------------------------------
% End of the pre-able, contents and lists of things
% Begin the Dedication page

\setstretch{1.3}  % Return the line spacing back to 1.3

\pagestyle{empty}  % Page style needs to be empty for this page
\dedicatory{For/Dedicated to/To my\ldots}

\addtocontents{toc}{\vspace{2em}}  % Add a gap in the Contents, for aesthetics


%% ----------------------------------------------------------------
\mainmatter	  % Begin normal, numeric (1,2,3...) page numbering
\pagestyle{fancy}  % Return the page headers back to the "fancy" style

% Include the chapters of the thesis, as separate files
% Just uncomment the lines as you write the chapters

\chapter{Introduction}

Tendencies shows that by introducing machine learning to extract informations from otherwise uncontrollable big data results in a leap-forward in main fields of marketing, media and researches. 
The key behind such efficiency is pattern recognition in high dimension spaces. 
When it comes to developement of architectures for finding optimal solutions usually the system is compared to human performance in cases which we think are easily solved. 
For better understanding the mechanism, furthermore the flaws of instances of artificial intelligence, the best is to disassemble the core idea to building blocks in a \textsc{top-down} manner, analyze each component separatively, and (re-)implement it step-by-step. 
After doing so, revealing the processes behind the architecture will be more intuitive than approaching the paradigm as a black box.

\section{A Section}

Quisque tristique urna in lorem laoreet at laoreet quam congue. Donec dolor turpis, blandit non imperdiet aliquet, blandit et felis. In lorem nisi, pretium sit amet vestibulum sed, tempus et sem. Proin non ante turpis. Nulla imperdiet fringilla convallis. Vivamus vel bibendum nisl. Pellentesque justo lectus, molestie vel luctus sed, lobortis in libero. Nulla facilisi. Aliquam erat volutpat. Suspendisse vitae nunc nunc. Sed aliquet est suscipit sapien rhoncus non adipiscing nibh consequat. Aliquam metus urna, faucibus eu vulputate non, luctus eu justo.

\subsection{A Subsection}

Donec urna leo, vulputate vitae porta eu, vehicula blandit libero. Phasellus eget massa et leo condimentum mollis. Nullam molestie, justo at pellentesque vulputate, sapien velit ornare diam, nec gravida lacus augue non diam. Integer mattis lacus id libero ultrices sit amet mollis neque molestie. Integer ut leo eget mi volutpat congue. Vivamus sodales, turpis id venenatis placerat, tellus purus adipiscing magna, eu aliquam nibh dolor id nibh. Pellentesque habitant morbi tristique senectus et netus et malesuada fames ac turpis egestas. Sed cursus convallis quam nec vehicula. Sed vulputate neque eget odio fringilla ac sodales urna feugiat.

\section{Another Section}

Phasellus nisi quam, volutpat non ullamcorper eget, congue fringilla leo. Cras et erat et nibh placerat commodo id ornare est. Nulla facilisi. Aenean pulvinar scelerisque eros eget interdum. Nunc pulvinar magna ut felis varius in hendrerit dolor accumsan. Nunc pellentesque magna quis magna bibendum non laoreet erat tincidunt. Nulla facilisi.

Duis eget massa sem, gravida interdum ipsum. Nulla nunc nisl, hendrerit sit amet commodo vel, varius id tellus. Lorem ipsum dolor sit amet, consectetur adipiscing elit. Nunc ac dolor est. Suspendisse ultrices tincidunt metus eget accumsan. Nullam facilisis, justo vitae convallis sollicitudin, eros augue malesuada metus, nec sagittis diam nibh ut sapien. Duis blandit lectus vitae lorem aliquam nec euismod nisi volutpat. Vestibulum ornare dictum tortor, at faucibus justo tempor non. Nulla facilisi. Cras non massa nunc, eget euismod purus. Nunc metus ipsum, euismod a consectetur vel, hendrerit nec nunc.
 % Introduction

%\input{Chapters/Chapter2} % Background Theory 

%\input{Chapters/Chapter3} % Experimental Setup

%\input{Chapters/Chapter4} % Experiment 1

%\input{Chapters/Chapter5} % Experiment 2

%\input{Chapters/Chapter6} % Results and Discussion

%\input{Chapters/Chapter7} % Conclusion

%% ----------------------------------------------------------------
% Now begin the Appendices, including them as separate files

\addtocontents{toc}{\vspace{2em}} % Add a gap in the Contents, for aesthetics

\appendix % Cue to tell LaTeX that the following 'chapters' are Appendices

\chapter{An Appendix}

Lorem ipsum dolor sit amet, consectetur adipiscing elit. Vivamus at pulvinar nisi. Phasellus hendrerit, diam placerat interdum iaculis, mauris justo cursus risus, in viverra purus eros at ligula. Ut metus justo, consequat a tristique posuere, laoreet nec nibh. Etiam et scelerisque mauris. Phasellus vel massa magna. Ut non neque id tortor pharetra bibendum vitae sit amet nisi. Duis nec quam quam, sed euismod justo. Pellentesque eu tellus vitae ante tempus malesuada. Nunc accumsan, quam in congue consequat, lectus lectus dapibus erat, id aliquet urna neque at massa. Nulla facilisi. Morbi ullamcorper eleifend posuere. Donec libero leo, faucibus nec bibendum at, mattis et urna. Proin consectetur, nunc ut imperdiet lobortis, magna neque tincidunt lectus, id iaculis nisi justo id nibh. Pellentesque vel sem in erat vulputate faucibus molestie ut lorem.

Quisque tristique urna in lorem laoreet at laoreet quam congue. Donec dolor turpis, blandit non imperdiet aliquet, blandit et felis. In lorem nisi, pretium sit amet vestibulum sed, tempus et sem. Proin non ante turpis. Nulla imperdiet fringilla convallis. Vivamus vel bibendum nisl. Pellentesque justo lectus, molestie vel luctus sed, lobortis in libero. Nulla facilisi. Aliquam erat volutpat. Suspendisse vitae nunc nunc. Sed aliquet est suscipit sapien rhoncus non adipiscing nibh consequat. Aliquam metus urna, faucibus eu vulputate non, luctus eu justo.

Donec urna leo, vulputate vitae porta eu, vehicula blandit libero. Phasellus eget massa et leo condimentum mollis. Nullam molestie, justo at pellentesque vulputate, sapien velit ornare diam, nec gravida lacus augue non diam. Integer mattis lacus id libero ultrices sit amet mollis neque molestie. Integer ut leo eget mi volutpat congue. Vivamus sodales, turpis id venenatis placerat, tellus purus adipiscing magna, eu aliquam nibh dolor id nibh. Pellentesque habitant morbi tristique senectus et netus et malesuada fames ac turpis egestas. Sed cursus convallis quam nec vehicula. Sed vulputate neque eget odio fringilla ac sodales urna feugiat.

Phasellus nisi quam, volutpat non ullamcorper eget, congue fringilla leo. Cras et erat et nibh placerat commodo id ornare est. Nulla facilisi. Aenean pulvinar scelerisque eros eget interdum. Nunc pulvinar magna ut felis varius in hendrerit dolor accumsan. Nunc pellentesque magna quis magna bibendum non laoreet erat tincidunt. Nulla facilisi.

Duis eget massa sem, gravida interdum ipsum. Nulla nunc nisl, hendrerit sit amet commodo vel, varius id tellus. Lorem ipsum dolor sit amet, consectetur adipiscing elit. Nunc ac dolor est. Suspendisse ultrices tincidunt metus eget accumsan. Nullam facilisis, justo vitae convallis sollicitudin, eros augue malesuada metus, nec sagittis diam nibh ut sapien. Duis blandit lectus vitae lorem aliquam nec euismod nisi volutpat. Vestibulum ornare dictum tortor, at faucibus justo tempor non. Nulla facilisi. Cras non massa nunc, eget euismod purus. Nunc metus ipsum, euismod a consectetur vel, hendrerit nec nunc.	% Appendix Title

%\input{Appendices/AppendixB} % Appendix Title

%\input{Appendices/AppendixC} % Appendix Title

\addtocontents{toc}{\vspace{2em}}  % Add a gap in the Contents, for aesthetics
\backmatter

%% ----------------------------------------------------------------
\label{Bibliography}
\lhead{\emph{Bibliography}}  % Change the left side page header to "Bibliography"
\bibliographystyle{unsrtnat}  % Use the "unsrtnat" BibTeX style for formatting the Bibliography
\bibliography{Bibliography}  % The references (bibliography) information are stored in the file named "Bibliography.bib"

\end{document}  % The End
%% ----------------------------------------------------------------