\documentclass{scrartcl}
\usepackage[magyar]{babel}
\usepackage[utf8]{inputenc}
\usepackage[T1]{fontenc}
\usepackage{graphicx, amsthm, lmodern, hologo}
\usepackage[export]{adjustbox}
\usepackage{listings}
\usepackage[colorlinks, linkcolor=blue, unicode, pdfusetitle]{hyperref}
\usepackage{bookmark}

\title{\LaTeX\ sablon szakdolgozathoz}
\subtitle{Használati utasítás}
\author{Marczell Márton}
\titlehead{\includegraphics[valign=m]{ITK_logo} \parbox[c]{\textwidth}{Pázmány Péter Katolikus Egyetem\\Információs Technológiai és Bionikai Kar}}

\newtheoremstyle{fineprint}{}{}{\footnotesize}{}{\itshape}{.}{ }{}
\theoremstyle{fineprint}
\newtheorem*{explanation}{Magyarázat}
\lstset{basicstyle=\ttfamily, columns=fullflexible}
\lstMakeShortInline{|}

\begin{document}
\maketitle
Ebbe a használati utasításba nem fér bele a \LaTeX\ használatának oktatása, csak néhány tanács.

\section{Követelmények}
Használat előtt győződj meg róla, hogy a \TeX\ disztribúciód (általában MiK\TeX\ vagy \TeX\ Live) a lehető legfrissebb, és rendelkezel az include-olt csomagok és a KOMA-Script documentclassok friss verzióival. Ezen felül szükséged lesz a Biberre, ami pedig egy külön program a bibliográfia generálásához. A disztribúció (vagy néha az operációs rendszer) csomagkezelőjével jellemzően ezt is egyszerűen fel lehet telepíteni. Online LaTeX szerkesztők esetén ezek elvileg gyárilag megvannak.

Ha a bibliográfia angolul van, a magyarításához töltsd le a |magyar.lbx| fájlt innen: \url{https://bitbucket.org/marczellm/latex-magyar-contrib/src}. Ha online LaTeX szerkesztőt használsz, akkor sajnos lehet, hogy manuálisan kell felraknod a magyar nyelvi csomagot is: \url{http://tug.ctan.org/language/hungarian/babel/magyar.ldf}

\section{A \LaTeX\ futtatása}
Így készíts PDF-et a forráskódból, ha a választott szerkesztőprogram (\TeX{Maker}, Kile stb.) ezt nem automatizálja. Parancssorban, a |bscthesis.tex| fájl mappájában futtasd ezt:
\begin{lstlisting}
pdflatex bscthesis.tex
biber bscthesis
pdflatex bscthesis.tex
pdflatex bscthesis.tex
\end{lstlisting}

\begin{explanation}
Az első |pdflatex| hívás végigmegy a dokumentumon, és segédfájlba írja a kereszthivatkozásokat és szakirodalmi hivatkozásokat. A |biber| beolvassa a |bib| fájlból a forrásjegyzéket, és újabb segédfájlba írja a ténylegesen meghivatkozott forrásokból összeállított, formázott bibliográfiát. A következő két |pdflatex| menetre a layout stabilizálása és az összes kereszthivatkozás feloldása miatt van szükség. Természetesen munka közben egyetlen |pdflatex| futtatás is elég egy nem végleges PDF elkészítéséhez.
\end{explanation}

\section{Hasznos tanácsok, gyakori tévhitek}
Ide olyasmiket írtam le, amit tapasztalatom szerint kevesen tudnak (vagy sokan rosszul), pedig fontos.
\subsection{Szóközök}
Olyan rövidítés után, ami nem végig nagybetűs, a pont miatt a \LaTeX\ azt hiszi majd, mondat vége van, és nagyobb szóközt rak. Normál szóköz lesz, ha elé írsz egy backspace-t:

\begin{lstlisting}
... mint pl.\ itt
\end{lstlisting}

Ha a rövidítés végig nagybetűs, de a mondat végén áll, |\@| használandó a pont előtt:

\begin{lstlisting}
ezt tartalmazza a PDF\@.
\end{lstlisting}

\subsection{Ábrák és táblázatok}
Tudományos szövegben az ábrák és táblázatok nincsenek fix helyre beszúrva a szövegbe, tehát nincs olyan, hogy ,,\ldots amint az az alábbi ábrán látható:'', hanem ,,\ldots amint az a 2.1.\ ábrán látható''. A |table| és |figure| környezeteknek \emph{éppen az a lényege}, hogy az ábra ne ott jelenjen meg, ahol te gondolod, hanem azt majd a \LaTeX\ jobban tudja. (Ezek úgynevezett floatok.) A \LaTeX\-nek lehet tippeket adni, hogy hova tegye ezeket:

\begin{lstlisting}
\begin{figure}[htbp]
\end{lstlisting}

A következő tippeket (engedélyeket) adhatod, érdemes többet is:
\begin{description}
\item[h] Futó szövegbe beszúrás
\item[t] Elhelyezés egy oldal tetején
\item[b] Elhelyezés egy oldal alján
\item[p] Elhelyezés külön oldalon
\end{description}

Így hivatkozz az ábrára:

\begin{lstlisting}
\begin{figure}
\centering
\includegraphics{kep.png}
\caption{informativ kepalairas}
\label{abraneve}  % ez csak belso hasznalatra
\end{figure}
\end{lstlisting}

majd a futó szövegben:

\begin{lstlisting}
amint az \az{\ref{abraneve}}.~abran lathato.
\end{lstlisting}

A |\az| parancsot a magyar Babel csomag definiálja, a megfelelő névelőt illeszti be a sorszám elé. A |~| nem törhető szóközt szúr be.

\subsection{Hivatkozások a felhasznált irodalomra}
Ajánlom figyelmedbe a |\cite|, |\textcite|, |\parencite| parancsokat a |biblatex| csomagból; részletek a csomag dokumentációjában.

\subsection{\LaTeX\ csomagok dokumentációja}
Interneten minden csomag friss dokumentációja megtalálható:

\begin{lstlisting}
http://texdoc.net/pkg/<csomag neve>
\end{lstlisting}

és parancssorból offline is, ha a csomag telepítve van a gépeden:
\begin{lstlisting}
texdoc <csomag neve>
\end{lstlisting}

\subsection{Egyebek}
Kódrészletekhez a |listings| vagy haladóknak a |minted|, táblázatokhoz a |booktabs|, vektorgrafikus ábrák készítéséhez (haladóknak) a |tikz| és |pgfplots| csomagokat ajánlom. Dokumentációkat el kell olvasni.

Az alapértelmezett betűtípust le lehet cserélni, ld. \url{http://www.tug.dk/FontCatalogue}. A gépeden telepített OpenType/TrueType betűtípusokat is lehet használni, ennek lépései:

\begin{enumerate}
\item Vedd ki az |inputenc| és |fontenc| csomagokat.
\item |\usepackage{fontspec}|
\item |\setmainfont{Cambria}| vagy amelyik tetszik
\item PDF\LaTeX\ helyett használj \hologo{XeLaTeX}-et vagy \hologo{LuaLaTeX}-et.
\end{enumerate}

\end{document}