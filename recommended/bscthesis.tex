%\documentclass[fontsize=11pt, appendixprefix=true]{scrreprt} -- ORIGINAL
\documentclass[10pt]{report}
\usepackage{tocloft}
%\renewcommand{\cftpartleader}{\cftdotfill{\cftdotsep}} % for parts
%\renewcommand{\cftchapleader}{\cftdotfill{\cftdotsep}} % for chapters
\renewcommand{\cftsecleader}{\cftdotfill{\cftdotsep}} % for sections, if you really want! (It is default in report and book class (So you may not need it).

% appendixprefix: hogy odaírja, hogy "Függelék A", ne csak "A"
%\usepackage[english, magyar]{babel}                        % nyelvi csomag
\usepackage[T1]{fontenc}                                   % ékezetes betűknél is legyen automatikus elválasztás
\usepackage[utf8]{inputenc}                                % ékezetes betűk kezelése
\usepackage{lmodern}                                       % alapértelmezett betűtípus ne legyen pixeles
\usepackage{mathtools}                                     % képletekhez kell
\usepackage[backend=biber, sorting=anyt]{biblatex}           % bibliográfia
\addbibresource{bscthesis.bib}

\usepackage{graphicx}                                      % képek beszúrása
\graphicspath{ {images/} }
\usepackage[export]{adjustbox}                             % ez az ITK logó pozicionálásához kell
\usepackage[margin=2.5cm, bindingoffset=1.25cm]{geometry}  % margók
\usepackage[onehalfspacing]{setspace}                      % másfeles sorköz
\usepackage[hidelinks, unicode, pdfusetitle]{hyperref}     % kattintható tartalomjegyzék és hivatkozások
\usepackage{bookmark}                                      % PDF könyvjelzők
\usepackage{csquotes}                                      % a bibliográfiában megfelelően legyenek formázva az idézőjelek
\DeclareQuoteAlias{german}{magyar}

% Kódrészletekhez ajánlom
\usepackage{listings, scrhack}
\usepackage{sourcecodepro} % egy jó betűtípus
\lstset{captionpos=b, numberbychapter=false, basicstyle=\ttfamily, showstringspaces=false, columns=fullflexible}
% Kódrészletek magyar stílusú számozása
\renewcommand\lstlistingname{kódrészlet}
\makeatletter
\renewcommand\fnum@lstlisting{\ifx\lst@@caption\@empty\else\thelstlisting.~\fi\lstlistingname}%
\makeatother
% Nyilatkozathoz két parancs definíciója
\newcommand{\pushtobottom}{\vspace*{\fill}}
\newcommand{\signatureline}[1]{\begin{flushright}
	\vspace*{.5cm}\par\noindent\makebox[2.5in]{\hrulefill}
	\par\noindent\makebox[2.5in][c]{#1}
	\end{flushright}
}


% Ezeket írd át!
\author{Csaba Botos}
\title{Building, testing and visualizing neural networks from scratch}
\date{2016}
%\subject{A thesis submitted for the Council of Scientific Students’ Associations}
%\publishers{Advisor:\\PhD.\ István Z. Reguly}

\usepackage{titlesec}

% Chapter customization
\titleformat{\chapter}[block]
  {\normalfont\Huge\bfseries}{\thechapter.}{1em}{\huge}

% For quotes
\usepackage{epigraph}
\setlength\epigraphwidth{0.7\textwidth}
\setlength\epigraphrule{0pt}

\begin{document}
%%%%%%%%%%%%%%%%%%%%%%%%%%%%%%%
%% BEVEZETÉS

\includegraphics[valign=m]{ITK_logo} \parbox[c]{\textwidth}{Pázmány Péter Catholic University\newline Faculty of Information Technology and Bionics}
\vspace*{\fill}

{\let\newpage\relax\maketitle}
\vspace*{\fill}
\begin{center}
A thesis submitted for the Council of Scientific Students’ Associations\\
Advisor: PhD.\ István Z. Reguly
\end{center}
\clearpage


\chapter*{Abstract}
\chapter*{Abstract}
Recently a special branch of Machine Learning, a model based on living organic systems called Deep Neural Networks is overtaking previous paradigm of algorithmic problem solving. 
It gained larger attention when better results were achieved than task-specific, handcrafted models in feature extraction. 
The reason behind its success is its scalability: the latest architectures are able to exploit the capacity of cutting-edge GPU hardware since the abstraction of the data is accomplished by succeeding neural nodes performing elementary operations, which can be easily paralellized.

It is of the utmost importance to understand the main concept of such networks to contribute to the breakthroughs of the fourth industrial revolution. 
To this purpose, building a framework from the base unit blocks of the newest models is the best introduction to Machine Learning. 
In my research I have disassembled black-box representated networks to the very basic, intuitive level and reorganized it in objectoriented manner, where each neural layer is treated as an entity derived from a common ancestor, therefore information flow and processes of the system are easily traced. 
My design and implementation is based on the principals of the components used by networks built for ImageNet classification, such as Convolutional, ReLU, Max-Pooling, Fully-Connected, Dropout, DropConnect, Softmax and k-Winner-Takes-All layers.
Furthermore, the following training methods and policies were adapted: cross validated, minibatch, on-line, $L_p$ regularized and basic Stochastic Gradient Descent training.

For testing the framework, the parameter space of Fully Connected networks was exhaustively explored. 
After training and evaluating sessions - mainly performed on the MNIST and self-acquired datasets - the results were gathered to analyze the performance of different architectures. 
For further investigation the best performing models  were compared to each other to find pros and cons of different capacity, layout and training of networks.

Besides architectural experiments, a non-trivial task targeted by many recent research of visualizing the inner representation of information, understanding transient activation patterns was studied as well.
Previously mentioned candidate networks were also visualized individually to retrieve information about characteristics of the processes in their Hidden Layers.
My implementation proposes a simplification of the DeconvNet derived from Gradient Ascent, an efficient algorithm to reveal patterns recognized by nodes in the hidden layers of Neural Networks, to produce adversarial input samples.
\addcontentsline{toc}{chapter}{Abstract}
\clearpage

\input{chapters/declaration}
\clearpage  % Declaration ended, now start a new page

\tableofcontents
\clearpage

\chapter{Introduction}
\section{Overview}


\epigraph{\textit{No one knows what the right algorithm is, but it gives us hope that if we can discover some crude approximation of whatever this algorithm is and implement it on a computer, that can help us make a lot of progress.}}{\rightline{{\rm --- Andrew Ng}}}

Every time we are interacting with our environment, we get closer to understand it. %still the closer we are the less we understand. 
However, as a byproduct, questions arise for which pure logical (mathematical?) solution is not available, or the problem is larger than to be solved by current ones.
Luckily we can always turn back to nature for inspiration: biological systems have proven their efficiency, therefore their functions worth to be further analyzed -- even if there could be theoretically a better way to tackle obstacles. 
One of the many branches of artificial intelligence, Neural Network is based on the nerve systems of living organisms which is capable of self-learning.

Such a simple paradigm is playing a fundamental role in boosting the industry and researches of today, because introducing Machine Learning to any field of life results in great leap forward. 
Thanks to the recent technological advancements, even with the current computational capacity of a personal computer one can get encouraging results by simply exploring the core principles of the topic.

\textbf{Main motivation.} Using outer libraries \cite{TF, torch, caffe}, without diving deep into mathematical proofs, treating neural networks as black-boxes thousands of useful applications \cite{haykin2004comprehensive} are made. 
On the other hand, building such architectures from the very basics helps to clarify how simple units might be organized, taught \cite{werbos1994roots}, and function \cite{hornik1989multilayer} as a large system. 
As complexity arises, the processes in the network becomes unclear and brings (yields?) the question: what is the purpose of each node? Methods to visualize how activation patterns formulate, and what information is held within them has been already investigated \cite{yosinski2015understanding}, and applied to improve performance \cite{zeiler2014visualizing}. My studies are mainly rely on recent publications, and researches in field of computer vision. The design of my own Deep Learning framework library is influenced by hot off press tutorials \cite{Goodfellow-et-al-2016-Book, deeplearningdotnet, nnsdl, stanfordlectures, gibiansky} and open-sources \cite{TF, torch, caffe} available for anyone. 

\textbf{Primary goal.} With my work I intend to bring closer, and demystify cutting-edge concepts of applied neural networkings for larger audiences. 
On related works and case studies I want to show what can be done by simply starting from the drawing board

\section{Importance of understanding}
\section{Thesis outline and contributions}






%
%\section*{Biological Motivation}
%\section*{Real Life Applications}
%\section*{Black Box representation}
%\section{The Perceptron model}
%\section{Importance of understanding}
%\section{Visualization}
	%2-5 oldal
\clearpage
\chapter{Literature}


% VÉGÉRE!
Considering course slides \cite{stanfordlectures, oxfordlectures} and textbooks \cite{Goodfellow-et-al-2016-Book, werbos1994roots} of acknowledged universities, popular websites \cite{deeplearningdotnet, pedregosa2011scikit}, blogs \cite{gibiansky, karpathyblog} and vlogs \cite{vlog1} on Artificial Neural Networks one might find hard to find a good point where to start. 
Some content may offer formal description of the general machine learning problem, others try to clarify through analogies with the biological nerve system.
Tutorials, which I have found interesting and the most helpful had several common features which are important to adapt, when designing a neural network library. 
Generally these common attributes are the following: 
they intend to be \emph{simple} as possible, 
have many \emph{intuitive} examples and analogies,
their \emph{interpretations} are not restricted to either universal approximators, or nerve systems of living organisms, but combines both aspect.		% minimum 10 oldal
\clearpage
%% BEVEZETÉS
%%%%%%%%%%%%%%%%%%%%%%%%%%%%%%%
%% TARTALOM
\input{chapters/designing}		% 
\clearpage
\input{chapters/implementation}	%
\clearpage
\input{chapters/results}		%
\clearpage
\input{chapters/conclusion}		%
\clearpage

%% TARTALOM
%%%%%%%%%%%%%%%%%%%%%%%%%%%%%%%
%% BEFEJEZÉS

\chapter{Summary}

In this work I provide a detailed introduction of building feed-forward neural networks, supported by both illustrative examples and mathematical proofs. 
The examples are based on biological motivations, and practical applications of NN. 
In the \textbf{Design and Implementation} chapter I present a guide to understanding and writing the neural network framework step-by-step, which I used to produce results in \textbf{Case Studies}. 
The newest implementation can be find at \cite{DV}, supported with out-of-the-box demo scripts, and \texttt{ipython notebooks}.
The library is modular in the first place, and easy to utilize because of the layer manager functions of the \texttt{network} class found in the \url{network_module}. 
It is also easy to \emph{fork} the implementation, and join the further development of the framework -- the phases of improvement are well documented, thanks to the version control supported by Git, and synchronized on the GitHub repository.

By working on this project I have understood the main concepts of Machine Learning, and experimented with multiple design patterns on how to realize the inference \ref{eq:forward} and backpropagation \ref{backprop} functions and arrived at the best fitting design: 
object-oriented implementation of \textbf{layers} handling the lattices of the computational graph derived from the abstract layer and \textbf{networks} managing the organization and training of layers.
My implementation offers both user-friendly interface and opportunity to fine-tune the network with advanced parameters as optional arguments.

I have utilized and tested the network on different tasks of popular Machine Learning problems, namely \textbf{Voice Recognition}, and \textbf{Image Classification}.
I was not just able to reproduce results of other researchers but I have also produced unique results;
while studying novel techniques of revealing perceptive fields and favored patterns of neurons, 
I have developed my own method for visualization by altering the backpropagation function of layers, arriving at the biased and unbiased \textbf{Gradient Ascent}.
I made experiments on the effects of GA, and studied how different shaped networks infer the input data, also how they could be exploited to make adversarial inputs. In a nutshell the results are the following: unbiased GA should be used for transforming samples into inputs which fools classifying networks, and biased GA for enhancing multiple samples to extract patterns from neural networks. The details of the experiments are explained in \textbf{\nameref{sec:MNIST}} section.
		%
\clearpage
\chapter{Future}

\paragraph{Short-term plans.} 
Currently I have a working version of the convolutional layer, 
however the implementation still relies on \texttt{convolve2d} function of ScyPy, which slows down the training process.
In the following months I will work on my own implementation of the convolution function, 
and improve the network usability.
Recently I have been able to train my network on CIFAR~\cite{cifar}, and ImageNET~\cite{deng2009imagenet} dataset, and the results are encouraging, however not enough yet to publish.
I am also working on \emph{Restrictive Boltzmann Machines} to be able to build and make experiments of \emph{Deep Belief} networks.

I am very interested in fields of \emph{Reinforcement Learning}, 
and I plan to utilize a network that plays the popular game, \texttt{agar-io}.
Also I am currently studying \emph{Recurrent Neural Networks} especially implementations of \emph{Long Short-Term Memory} architectures, 
which are not just better for audio recognition tasks, but can be used as Generative networks, producing artificial samples.
I want to create an application capable of accompanying musicians in jam-sessions, based on \emph{LSTM} networks.
I will continue my studies in field of \emph{Generative Adversarial Networks} which is currently a very hot topic of Computer Vision.

\paragraph{Long-term plans.}
My studies have two basic motivation:
When observing living organisms I think about how could we model them, 
and reverse-engineer their function. 
On the other hand, I want to get closer to understand our 
environment and how can we build knowledge based on our experiences.
I believe, that one day my studies in Neuroscience and Artifical Intelligence will converge to a common point, and I will be able to model cognitive functions of the human mind.
			%
\clearpage

\printbibliography
\addcontentsline{toc}{chapter}{Bibliography}
%% BEFEJEZÉS
%%%%%%%%%%%%%%%%%%%%%%%%%%%%%%%
\end{document}