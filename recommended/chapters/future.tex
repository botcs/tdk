\chapter{Future}

\paragraph{Short-term plans.} 
Currently I have a working version of the convolutional layer, 
however the implementation still relies on \texttt{convolve2d} function of ScyPy, which slows down the training process.
In the following months I will work on my own implementation of the convolution function, 
and improve the network usability.
Recently I have been able to train my network on CIFAR~\cite{cifar}, and ImageNET~\cite{deng2009imagenet} dataset, and the results are encouraging, however not enough yet to publish.
I am also working on \emph{Restrictive Boltzmann Machines} to be able to build and make experiments of \emph{Deep Belief} networks.

I am very interested in fields of \emph{Reinforcement Learning}, 
and I plan to utilize a network that plays the popular game, \texttt{agar-io}.
Also I am currently studying \emph{Recurrent Neural Networks} especially implementations of \emph{Long Short-Term Memory} architectures, 
which are not just better for audio recognition tasks, but can be used as Generative networks, producing artificial samples.
I want to create an application capable of accompanying musicians in jam-sessions, based on \emph{LSTM} networks.
I will continue my studies in field of \emph{Generative Adversarial Networks} which is currently a very hot topic of Computer Vision.

\paragraph{Long-term plans.}
My studies have two basic motivation:
When observing living organisms I think about how could we model them, 
and reverse-engineer their function.
I want to contribute with my own studies to the current level of Artificial Intelligence.
On the other hand, via visualization I want to get closer to understand our 
environment and how can we build knowledge based on our experiences.
