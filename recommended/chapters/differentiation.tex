\section{Differentiation}\label{sec:diff}
In the above evaluating of $\nabla_\phi \mathcal{F}$ can be done in two ways, namely by numerical approximation, or by analytical derivation, in the following I will discuss both.

\subsection{Numeric differentiation} Evaluating the numerical gradient (or difference) is an elementary, yet powerful operation, in which we would \emph{perturb}, or modify one parameter $\phi$ of our system $\mathcal{F}$ at once.
That is done by first adding $\phi^+$ and after subtracting $\phi^-$ a little amount $d\phi$ from the original $\phi$ and evaluate $\mathcal{L}^{\pm}=\mathcal{L}(\mathcal{F}_{\phi^{\pm}})$, namely the \emph{Loss} of the system in the modified state, yielding the numerical gradient in the following equation:

\begin{equation} \label{eq:numgrad}
    \frac{d\mathcal{L}(\mathcal{F})}{d\phi} = 
    \frac{\mathcal{L}^+ - \mathcal{L}^-}{2 d\phi} = 
    \frac{\mathcal{L}(\mathcal{F}_{\phi^+}) - \mathcal{L}(\mathcal{F}_{\phi^-})}{2 d\phi}
\end{equation}

\paragraph{Summary.} In a nutshell value of $\frac{d\mathcal{L}(\mathcal{F})}{d\phi}$ tells how changing the parameter $\phi$ by $d\phi$ would change the performance of the network. If it is positive then updating $\mathcal{F}$ by adding $d\phi$ to $\phi$ would result in higher Loss value, which is the opposite of our goal, so we just subtract it, if it is negative, than trivially we should add $d\phi$ to $\phi$ since it is making some good progress.

\subsection{Complexity} 
Though letting the computer do the hard work seems to be a good idea, it worths considering that the simple method above will be applied to every $\phi$ of $\mathcal{L}$.
It means that for the network in the toy example, we need to evaluate $\mathcal{L}(\mathcal{F})$ two times for each parameter in 
the weight matrices and the bias vectors of the network, totaling in 
$$\#(\phi) = 2\times\sum_{i=1}^3 N_{(i-1)}\cdot N_i + N_i = 2\times(10\cdot 5 + 5 ... + 4\cdot 3 + 3) = 188$$
Even if $\mathcal{F}$ is approximated by using $k$-sized mini-batches for evaluation it is still a computationally very expensive function, because the inference would result in the following number of operations of addition and multiplication:
$$\#(\mathrm{operations}) = k\times\sum_{i=1}^3 2\cdot N_{(i-1)}\cdot N_i + N_i = k \times 176$$
Therefore approximated with $k=10$ mini-batches would a single parameter update of a very tiny network would require total operations of:
\begin{equation}
    \#(\mathrm{total})=\#(\phi) \times k \times \#(\mathrm{operations}) = 188 \cdot 10 \cdot 176  = 330880
\end{equation}
Because both $\#(\phi)$ and $\#(\mathrm{operations})$ has complexity of $\mathcal{O}(N^2)$, one update will yield complexity of 
\begin{equation}
    \#(\mathrm{total})=\mathcal{O}(N^4)
\end{equation}
We can see that even for a shallow and relatively small network (industrial AI networks has billions of parameters, and uses much larger batches) described in the toy example the method is really costly.
That encourages us to derive our differentials on paper first, and use \emph{numerical gradient approximation} for checking our solution. Using \emph{Gradient Check} is essential when implementing new architectures, because it is a very efficient tool for debugging in comparison with updating. The method in a few words is about setting an error rate $\epsilon$, and decrease $d\phi$ until the numeric solution does not match the analytic solution with $1-\epsilon$ significance. If the analytic solution is incorrect the cycle will not terminate.

\subsection{Analytic differentiation}
Deriving the update by hand requires basic knowledge in calculus extended to multivariate cases, though since the operations are elementary, in general we must understand only three basic definitions to do so. 
Some formality before starting: we have three independent variables $x$, $y$, $z$ and functions $f$, $g$, $h$. The result of operations performed on variables, i.e. $x+2y$ can be represented by a function $f=x+2y$. 
If the value depends on a variable then it can be written explicitly, passing the variable as \emph{the argument} of the function $f(x,y)=x+2y$. 
For the sake of simplicity assume that the variables are not general objects from an abstract space, they are only real values: $x,y,z\in \mathbb{R}$. However the following description could be extended for the above-mentioned variables as well. For any one-dimensional function $f(x):\mathbb{R}\mapsto\mathbb{R}$ we say that the value represented by the function depends on the variable by the extent of its derivative. The derivative (or differential) of the function can be seen as an ideal case of \ref{eq:numgrad} where the perturbation would approach zero, namely:
\begin{equation}
    \frac{\partial f}{\partial x} = \lim_{dx\rightarrow 0}\frac{df(x)}{dx}= \lim_{dx\rightarrow 0} \frac{f(x+dx)-f(x-dx)}{2dx}
\end{equation}
\paragraph{Multiplication rule.}
Consider a value $x\cdot y \cdot z$ represented by $f(x,y,z)$. $f$ is now depending on three variables, we can define the measure of this dependency on one variable by the formal equation:
\begin{equation*}
\begin{split}
    \frac{\partial f}{\partial x} = \lim_{dx\rightarrow 0} \frac{f(x+dx,y,z)-f(x-dx,y,z)}{dx}
    &= \lim_{dx\rightarrow 0} \frac{((x+dx)\cdot y \cdot z)-((x-dx)\cdot y \cdot z)}{2dx} \\
    = \lim_{dx\rightarrow 0} \frac{(x+dx)-(x-dx)}{2dx} y z&= \lim_{dx\rightarrow 0} \frac{2dx}{2dx} yz= yz
\end{split}
\end{equation*}
We can apply the same method for each variable, the result will be the elements of the \emph{gradient} $\nabla f$
\begin{equation}\label{eq:multiplication}
    \frac{\partial f}{\partial x} = y z \qquad
    \frac{\partial f}{\partial y} = x z \qquad
    \frac{\partial f}{\partial z} = x y 
\end{equation} 
The important thing to understand that in a computational graph, a multiplicative node, which takes $N$ arbitrary parameters (or arguments), will have a \emph{partial derivative} for each variable its output is depending on. In general if these derivatives are represented as a vector, then it is called the gradient $\nabla f$ of $f$. Also the value of the derivative will be the product of all variables except the one of which we are computing the influence of on the output.

\paragraph{Addition rule.}
Consider a value $x\cdot y + x\cdot z$ represented by $g(x,y,z)$.
The change of $g$ with respect to $x$ is defined with the following shortened equation:
\begin{equation}\label{eq:addition}
    \frac{\partial g}{\partial x} = \lim_{dx\rightarrow 0} \frac{((x+dx)y +(x+dx)z)-((x-dx)y +(x-dx)z)}{dx}=y+z
\end{equation}
Notice that -- in the terms of computational graphs -- if a node contributes to other different operations (namely $x\cdot y$ and $x\cdot z$), 
than the derivative of each occurrence in \emph{later} values will be summed up. 

\paragraph{Chain rule.}
Let $f(x)=2x+3$ and $g(f)=5f$. Suppose that we would like to know the derivative of $g$ with respect to $x$.
At first we cannot do so, but there are two options: in the hope that substituting the value represented by $f$ into $g$ would not make the equation too complex we can unroll the references and rewrite $g(x)=5\cdot(2x + 3)$, or we could use the chain rule:
$$
    \frac{\partial g}{\partial x} = \lim_{dx \rightarrow 0} \frac{g(f(x+dx)) - g(f(x-dx))}{2dx}
$$

\begin{center}
    Assume that $f(x+dx)-f(x-dx) \neq 0$.
\end{center}

$$
    \lim_{dx \rightarrow 0} \frac{g(f(x+dx)) - g(f(x-dx) ) }{f(x+dx)-f(x-dx)} \cdot \frac{f(x+dx)-f(x-dx)}{2dx}
$$

\begin{equation}\label{eq:chain}
    \frac{\partial g}{\partial x} = \frac{\partial g}{\partial f} \cdot \frac{\partial f}{\partial x}
\end{equation}

Which is a formula of the products of partial derivative of $g$, that treats $f$ like a variable, and $f$ explicitly operating on variable $x$.
The derivative of $g$ with respect to variable $x$ is the product of the \emph{local derivative} of $g$ is $\frac{\partial g}{\partial f}=5$ and $\frac{\partial f}{\partial x}=2$ which equals $\frac{\partial g(x)}{\partial x}=\frac{\partial(5\cdot (2x + 3))}{\partial x} = 10$, the function strictly depending on $x$.
The important message is that we can interchangeably use function values and variables with a constraint that at a point, in an arbitrary depth there must be a real variable.
\emph{Note:} the statement above stands for computations with \emph{Acyclic Graph}, meaning that there should not be any feedbacks or loops -- no definitions like $f(g), g(f)$ or $f(f)$.
We will see that these rules play a very fundamental role in training networks.

\paragraph{Vector Notation.}
Before drilling deep into mathematical equations, a small reminder: the following vector and matrix formulation is just a special annotation, 
using the rules above, which helps to make clear both the definition, and the computations done by the network when it is implemented. 
The vectors with partially derivatives inside are just representing \emph{real values}, arranged in a fancy way.
Every vector and matrix defined in forward propagation, has its corresponding derivative w.r.t. the Loss.
More trivially, if any value is depending on a list of variables $f(x_1, x_2 \cdots x_n) = f(\mathbf{x})$ (a vector) then there is a list of \emph{partial derivatives} w.r.t. to $f$ -- forming the gradient $\nabla f = \left(\frac{\partial f}{x_1}, \frac{\partial f}{x_1} \cdots \frac{\partial f}{x_N}\right)$. 
\emph{Notation}: When the vector notation is emphasized the variable name is conventionally written in bold font $\mathbf{x}$, or is underlined \underline{$x$}.
Because later the indexing would become too crowded, we only use the indexed notation when it is necessary, otherwise using $x$.

The next step is formulating the dependency of multiple functions on multiple variables.
As seen above, a multivariate function has its gradient vector --
in the same fashion as the list of variables were organized into vectors $\mathbf{x}$, values composed of them can also form a vector $F = (f_1, f_2 \cdots f_M)$, composing a multivalued function depending on the same variables.
Doing so yields a first-order derivative matrix, composed of gradient vectors $J=(\nabla f_1, \nabla f_2 \cdots \nabla f_M)^T$, called the \emph{Jacobian of $F$}.
The Jacobian has as many rows as output values $F$ has, and the same number of columns of the variables that $f_i$ is a function of.
\begin{equation}\label{eq:jacobian}
    J(F) = 
    \begin{pmatrix}
    \quad \nabla f_1 \quad \\ 
    \vdots \\ 
    \nabla f_M
    \end{pmatrix} 
    =
    \begin{pmatrix}
    \frac{\partial f_1}{\partial x_1} & \cdots & \frac{\partial f_1}{\partial x_N} \\ 
    \vdots & \ddots & \vdots \\ 
    \frac{\partial f_M}{\partial x_1} & \cdots & \frac{\partial f_M}{\partial x_N}
    \end{pmatrix} 
\end{equation}
The matrix and vector operations (such as addition and inner product) that can be performed on the derivative \emph{arrays} are identical defined in the inference section. That is important because product of derivatives introduced by the chain rule, can be applied as well for multidimensional array of derivatives too.

\subsection{Fully Connected Layer}
Return to the toy example and begin with the last layer, with $3$ nodes. If a sample $x$ is inferred $\mathcal{F}(x)=y$, then the response of the network will be a vector of $dim(y) = 3$. If we took the $L_2$ \emph{distance} between the response and the goal $y^*$, then it would tell how far we are from the ideal, by a single scalar value. Since we want to minimize it, we have to adjust the parameters of the network, namely descend on the gradient slope. To get the small extent of the update we have to evaluate $\frac{\partial\mathcal{L}}{\partial \phi}$. 
Intuitively in the case we would like to correct the weights of a decision, it would require two things:
\begin{itemize}
    \item[] The original situation (the input of the $l^{th}$ layer $x_l$), which the decision was made in.
    \item[] The error on the decision -- the derivative $\delta^l$ of the Loss with regards to the decision.
\end{itemize}
Since $x_l$ is obtained via inference, what we have to calculate is $\delta^l$ for the $l^{th}$ layer in order to acquire the parameter gradient.
The first step is to evaluate the $\delta_L$, or the \emph{error} of the last layer's response $y^3$, namely $\delta^3 = \nabla_{y^3} \mathcal{L}$.
Begin with the first element: 
$$
    \delta_1^3 = 
    \frac{\partial \mathcal{L}}{\partial y_1} = 
    \frac{\partial}{\partial y_1}\frac{1}{2}\left((y^*_1 - y_1)^2 + (y^*_2 - y_2)^2 + (y^*_3 - y_3)^2\right) = y_1 - y^*_1
$$
\begin{center}
Expanding it to the whole array:
\end{center}
$$
    \delta^3 = \begin{pmatrix}
     \frac{\partial \mathcal{L}}{\partial y_1}\\ \\
    \frac{\partial \mathcal{L}}{\partial y_2} \\ \\
    \frac{\partial \mathcal{L}}{\partial y_3}
    \end{pmatrix} = \begin{pmatrix}
     \frac{\partial}{\partial y_1} \frac{1}{2}\sum_i(y_i^*-y_i)^2\\ \\
    \frac{\partial}{\partial y_2} \frac{1}{2}\sum_i(y_i^*-y_i)^2 \\ \\
    \frac{\partial}{\partial y^3_3}  \frac{1}{2}\sum_i(y_i^*-y_i)^2
    \end{pmatrix} = \begin{pmatrix}
     {y_1-y^*_1}\\ \\
     {y_2-y^*_2}\\ \\
     {y_3-y^*_3}
    \end{pmatrix} 
$$
Consider the following notation:
$
    \delta^3_i = (y^*_i - y_i)
$, 
called parametric vector notation. Writing arrays in this way, saves a lot of space. However, when this notation gets jammed with indexes, it is useful to write down explicitly the whole array for clarification.

\paragraph{The last layer} Now we have exact values of $\delta^3$ and $x^3$, so we can calculate how should the weights in $W^3$ be changed in order to get a better network.
Applying the differentiation rules for each weight (forming a matrix) of the layer will result in a derivative for each weight (also forming a matrix).
Taking the first perceptron of the layer, it has a weight $W^3_1=(W_{1,1}, W_{1,2}, W_{1,3}, W_{1,4})$ for each output of the previous layer.

Suppose that this neuron had to tell how rounded is the object on an image sample, 
and the $i^{th}$ neuron of the $2^{nd}$ layer fires when it recognizes sharp edges.
Of course it would be bad if our neuron had a large weight on $x^3_i$. 
If this node performs poorly because of $x^3_i$, then it would contribute a lot to the Loss function while inferring sharp objects, with its output $y_1$ being far away from $y^*_1$, resulting in a positive $\delta^3_1$. 
In case of $x^3_i = 0$, the weight $W_{1,i}$ has nothing to do with the error $\delta^3_1$ of the neuron.
Notice that the error of each weight (w.r.t $\mathcal{L}$) should be proportional to the error and the input as well: $\frac{\partial \mathcal{L}}{\partial w_{1,i}}=x^3_i \cdot \delta^3_1$.
However it can be also derived in terms of the differentiation rules: 
$$
    \frac{\partial \mathcal{L}}{\partial W_{1,i}}=
    \frac{\partial \mathcal{L}}{\partial y^3_1}\cdot \frac{\partial y^3_1}{\partial W_{1,i}} =
     \delta^3_1  \cdot x^3_i
$$

\paragraph{Considering the parameter update.} assume that a picture of an origami sculpture (a very edgy one) was inferred and the $i^{th}$ node of the second layer worked correctly. Both $x^3_i$ and $\delta^3_1$ is positive, however the weight should be decreased: that is why we will take the negative of the derivative for updating $w_{1,i}$.
Substituting $i=1,2,3,4$ into $\frac{\partial \mathcal{L}}{\partial w_{1,i}}$ yields a gradient of $\mathcal{L}$ with regards to the weights of the first neuron of the last layer. 
Doing so for each neurons in the layer would result in 3 gradient vectors $\nabla_{W_1} \mathcal{L}$, $\nabla_{W_2} \mathcal{L}$ and $\nabla_{W_3 }\mathcal{L}$ which is practically stacked to make a matrix which can be later added element-wisely to the weight matrix $W$.
$$
\nabla_W \mathcal{L} = 
\begin{pmatrix}
\frac{\partial \mathcal{L}}{\partial W_{1,1}} & \frac{\partial \mathcal{L}}{\partial W_{1,2}} & \frac{\partial \mathcal{L}}{\partial W_{1,3}} & \frac{\partial \mathcal{L}}{\partial W_{1,4}} \\ \\
\frac{\partial \mathcal{L}}{\partial W_{2,1}} & \frac{\partial \mathcal{L}}{\partial W_{2,2}} & \frac{\partial \mathcal{L}}{\partial W_{2,3}} & \frac{\partial \mathcal{L}}{\partial W_{2,4}} \\ \\
\frac{\partial \mathcal{L}}{\partial W_{3,1}} & \frac{\partial \mathcal{L}}{\partial W_{3,2}} & \frac{\partial \mathcal{L}}{\partial W_{3,3}} & \frac{\partial \mathcal{L}}{\partial W_{3,4}}
\end{pmatrix} = 
\begin{pmatrix}
 x_1 \delta_1 &  x_2 \delta_1 &  x_3 \delta_1 &  x_4 \delta_1 \\ \\
 x_1 \delta_2 &  x_2 \delta_2 &  x_3 \delta_2 &  x_4 \delta_2 \\ \\
 x_1 \delta_3 &  x_2 \delta_3 &  x_3 \delta_3 &  x_4 \delta_3
\end{pmatrix}
$$

The last part of the equation can be also expressed as:
\begin{equation}
\nabla_W \mathcal{L} = \left(\frac{\partial \mathcal{L}}{\partial W_{i,j}}\right) = \left(x_j \delta_i\right) = x \wedge \delta
\end{equation}
Where $\wedge$ denotes the \emph{outer product} operator. 
The gradient of the bias is simply $\nabla_{b^L} \mathcal{L} = \delta^L$.

\paragraph{The $(L-1)^{th}$ layer.} Gradient descent can be applied on networks with more than one layer, 
however continuing the example of the image descriptor network requires a bit more abstraction. 
In the previous explanation we assumed that the $i^{th}$ neuron of the second layer worked properly.
In general cases this assumption is incorrect, if the whole network is initialized at once.
If we think about that also the mentioned neuron in the \emph{last hidden layer} should be trained, 
then we can apply the same method with a 4 dimensional $\delta^{(L-1)}$ and with a 5 dimensional $x^{(L-1)}$ input.

\emph{Note}: Capital $L$ is representing the number of layers. In the toy example $L = 3$.\\
Acquiring the $\delta^{(L-1)}$ is where the chain rule \eqref{eq:chain} steps into the scene.
While in the example before we could find an intuitive workaround, in this case it would be quite strained, since $\mathcal{L}$ does not depend directly on $y_{(L-1)}$.
Utilizing the chain rule:
\begin{equation*}
    \delta^{(L-1)} = 
    \frac{\partial \mathcal{L}}{\partial y^{(L-1)}} = 
    \frac{\partial \mathcal{L}}{\partial y^{L}} \cdot
    \frac{\partial y^{L}}{\partial y^{(L-1)}} = 
    \delta^L \cdot J(F_L)
\end{equation*}
Where $J(F_L)$ denotes the \emph{Jacobian} \eqref{eq:jacobian} of the function representing the projection of the last layer $F_L:\mathbb{R}^4\mapsto \mathbb{R}^3$.
It is a map of how each input variable affects the output of the layer.
In general: the Jacobian numerically can be represented as a $3 \times 4$ matrix, analytically as a \emph{local derivative} of function $F$.
In case of \emph{Fully Connected} layers the Jacobian is simply the weight matrix $F(F_l)=W_l$ (the bias drops out here).

For derivative of scalar values ($\mathbb{R}$) the \emph{product} operator is well defined, 
and it can be expanded the same way to derivatives of multidimensional values as 
regular \emph{inner product} of the values they are composed of.
\emph{Note}: In order to stay consistent with the dimensions of the computation, we have to switch sides of the matrix multiplication defined in \eqref{eq:FC}:
\begin{equation}\label{eq:reverse}
\begin{split}
\delta^{(L-1)}_j &= 
    \sum_j \delta^L_i \; W^L_{i,j} \\
    \delta^{(L-1)} &= \delta^L \cdot W^L\\
    \left[4\right] &= \left[3\right] \cdot \left[3 \times 4\right] 
\end{split}
\end{equation}
\subsection{Backpropagation}
The backpropagation algorithm, first described by Werbos \emph{et. al} \cite{werbos1994roots}
\paragraph{The $l^{th}$ layer.} 
The contribution to the Loss of the general $l^{th}$ layer $\delta^l$ can be retrieved by unfolding $\frac{\partial \mathcal{L}}{\partial y^l}$ applying the chain rule, namely the backpropagation:
\begin{equation}\label{backprop}
\begin{split}
    \delta^l &= 
    \frac{\partial \mathcal{L}}{\partial y^{l}} = 
    \frac{\partial \mathcal{L}}{\partial y^{L}} \cdot
    \frac{\partial y^{L}}{\partial y^{(L-1)}} \quad \cdots \quad
    \frac{\partial y^{(l+1)}}{\partial y^{l}} \\
    \delta^l &= \delta^L \cdot J(F_L) \cdot J(F_{(L-1)}) \quad \cdots \quad J(F_{(l-1)})\\
    \left[dim(l)\right] &= \left[dim(L)\right] \cdot \left[dim(L)\times dim(L-1)\right] \cdots \left[dim(l+1)\times dim(l)\right]
\end{split}   
\end{equation}
\emph{Note}: the order of evaluating these derivatives is theoretically irrelevant, 
however computationally there is an opportunity to implement it in two ways \cite{akthesis}:
\begin{enumerate}
    \item[] \emph{forward-mode differentiation}: evaluating in the order of layers is efficient in cases where the output of the network is much larger than the input
    \item[] \emph{reverse-mode differentiation}: evaluating in reversed order for networks with fewer outputs than inputs.
\end{enumerate}
As pointed out in the thesis, the former would require $(L-l)$  times evaluating a \emph{matrix-matrix} product and one \emph{vector-matrix} operation,
while the latter would require $(L-l)$ times evaluating a \emph{vector-matrix} product and one \emph{matrix-matrix} at the end. 
Using forward-mode differentiation does not require keeping transient activations $y^l$, however it is computationally costly.
Using backward-mode differentiation does not strain the CPU, but the memory. 
It is because if $J(F_l)$ is not linear, then it requires the input $x_l$ to evaluate the first order derivatives.

\subsection{Activation Layer} 
Though the activation layer operates on the input, it has no adjustable parameter. 
Since we have to evaluate $\delta$ for layers behind activation layers, the error must pass through $J(F_{activation}$ as well.
Luckily activation layers applies a scalar function element-wise on the inputs, 
the Jacobian of the composite function $F$ has a special attribute: it is \textbf{diagonal}, meaning that:
$$
    J(F) = \frac{\partial F_i}{\partial x_j} = 
    \begin{cases}
        \frac{\partial f(x_i)}{\partial x_i} & i = j\\
        0 & i \neq j
    \end{cases}
$$
We can exploit this function when implementing activation layers, by simply using element-wise product $\frac{\partial f(x_i)}{\partial x_i}$ instead of a matrix multiplication.
The mentioned activation functions (\ref{eq:af1}, \ref{eq:af2}, \ref{eq:af3}, \ref{eq:af4}), are differentiable \emph{almost everywhere}, the corresponding derivatives:
\begin{align}
    \mathrm{(ReLU)' := } &\mathrm{Heaviside}(x)\\
    \mathrm{(TanH)' := }   & 1 - \tanh(x)^2\\
    \mathrm{(SP)' := }   &\frac{1}{1+e^{-x}}\\
    \mathrm{(Log)' := }  &\mathrm{Logistic}(x)\cdot (1-\mathrm{Logistic}(x))
\end{align}
