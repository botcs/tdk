\chapter{Literature}


% VÉGÉRE!
Considering course slides \cite{stanfordlectures, oxfordlectures} and textbooks \cite{Goodfellow-et-al-2016-Book, werbos1994roots, bengio2009learning} of established universities, popular websites \cite{deeplearningdotnet, pedregosa2011scikit}, articles \cite{lecun2015deep}, blogs \cite{gibiansky, karpathyblog} and vlogs \cite{vlog1} on Artificial Neural Networks one might find it hard to find a good point to start. 
Some content may offer formal description of the general machine learning problem, others try to clarify through analogies with the biological nerve system.
Tutorials, which I have found interesting, and the most helpful, had several common features which are important to adapt, when designing a neural network library. 
Generally these common attributes are the following: 
they intend to be \emph{simple} as possible, 
have many \emph{intuitive} examples and analogies,
their \emph{interpretations} are not restricted to either universal approximators, or nervous systems of living organisms, but combines both aspect.