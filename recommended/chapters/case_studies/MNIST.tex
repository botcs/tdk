
\section{Handwritten digit recognizing}
First layer can be easily visualized by just simply reshaping the weights, but extracting patterns which are recognized by further layers are not trivial. The main goal is to find an algorithm which with the input images can be enhanced to highlight the patterns that causes strongest activation in each neuron.


\subsection{Training}
\label{train}
The networks were trained on the MNIST dataset \cite{mnist}, with varying the following hyperparameters:
\begin{itemize}
    \item Number of total neurons, with different distribution between layers
    \item Number of hidden layers
    \item Learning rate - constant during epochs
    \item Regularization - $L_p$
\end{itemize}

\begin{figure}
    \centering
    \includegraphics[width=0.9\textwidth]{shapeNN-even.png}
    \caption{Evenly distributed networks don't the input, even the last layers tend to activate on various input patterns. During backpropagation the gradient is also more likely to disappear.}
    \label{fig:even}
\end{figure}
\begin{figure}
    \centering
    \includegraphics[width=0.9\textwidth]{shapeNN-dec.png}
    \caption{Decreasing networks tend to compress data through layer to layer, highest result was achieved with decreasing width network}
    \label{fig:dec}
\end{figure}
\begin{figure}
    \centering
    \includegraphics[width=0.9\textwidth]{shapeNN-inc.png}
    \caption{During forward propagation the data is compressed in early layers, the later neurons rely on activation patterns of the coding layer. Networks which width of layer increase in order tends learn slower, the process is inefficient, since information is lost in the beginning.}
    \label{fig:inc}
\end{figure}


The experiments width shapes and corresponding results are shown on \emph{Figure (\ref{fig:even}, \ref{fig:dec}, \ref{fig:inc})}.
The networks were trained with the following constraints: Stochastic gradient descent with $mini-batch=64$, learning rate $\lambda = 0.05$, no regularization, total number of $epochs=25$.
Despite the varying capacity, the results rather depend on the shape of the network. The best of the three networks are compared in \emph{Figure \ref{fig:comp}} 

\begin{figure}
    \centering
    \includegraphics[width=0.6\textwidth]{comp.png}
    \caption{Comparison of the three type of distribution of neurons between layers}
    \label{fig:comp}
\end{figure}


\subsection{Candidates for enhancing}
\label{method}
For a given network and set of input samples, a set of candidates can be found, which yield the highest output w.r.t. each neuron. 
For a given layer, the search can be done parallel, and for more general results, not only the best samples are taken into account, but the top $T$ inputs yielding the highest activation of all samples from the given search set.
\begin{figure}
    \centering
    \includegraphics[width=0.6\textwidth]{NN1.png}
    \caption{First step of visualizing the $l^{th}$ layer: selecting each neuron's most confident choice from the image set}
    \label{fig:n1}
\end{figure}

\subsection{Enhancing - gradient ascent}

Simply mean averaging the candidates would not reveal the true signals which each neurons look for, see Figure \ref{fig:mean-ga-comp}

\begin{figure}
    \centering
    \includegraphics[width=0.6\textwidth]{mean-ga-comparison.png}
    \caption{On the first set of pictures the mean of the $T=100$ candidates are shown, they are the most preferred inputs of the candidate set for the last layer's neurons. On the set depicted below the enhanced images are shown}
    \label{fig:mean-ga-comp}
\end{figure}
\begin{figure}
    \centering
    \includegraphics[width=0.8\textwidth]{NN2.png}
    \caption{One iteration shown: a hidden layer's weights cannot be reshaped into images, but applying their activation gradient on the input images will emphasize, and amplify the patterns which yields the highest output w.r.t each neuron}
    \label{fig:n2}
\end{figure}
Gradient ascent is an iterative method for enhancing the input to emphasize the patterns that are recognized by the neurons. A method which utilizes a network on top of the original to extract the recognition patterns, called DeConvnet, is described by Zeiler \emph{et al.} \cite{zeiler2014visualizing}. The mentioned algorithm can be reduced to the following, by leaving out the sparsity encouraging regularization term - making parallel processing more efficient:
for fully connected layers the method boils down to backward propagating an \emph{arbitrary} delta vector - instead of calculating the gradient of the loss function described in \cite{zeiler2014visualizing}:
$$
\delta_l = \mathbf{e_{i}}
$$
from the $l^{th}$ layer toward the input layer to get gradient of the input w.r.t the $i^{th}$ neuron of the $l^{th}$ layer. 

By choosing $\mathbf{e_{i}} = [0, 0 \cdots 1 \cdots 0]^T$ the enhancement would not take other activations into account, while letting $\mathbf{e_i} = [-1, -1 \cdots 1 \cdots -1]^T$ would result in activation where other neurons' output would decrease over amplification. For further analysis $\mathbf{e_i} = [0, 0 \cdots 1 \cdots 0]^T$ is used.
$$x = x + \mathbf{\eta}\cdot\nabla_l^i x$$
Applying the gradient to the input $x$ is the first step of the iteration (see Figure \ref{fig:n2}). Next step is forwarding the enhanced input, applying the $\delta$ on the $l^{th}$ layer and backwarding, and applying it on the input again makes a full cycle.
The hyperparameters of the method are the \emph{rate of amplification $\mathbf{\eta}$} and \emph{number of iterations $I$}

\subsection{Processing enhanced samples}
The results of amplifying the top $T$ samples for $n_l$ number of neurons in the $l^{th}$ layer are stored in a $d + 2$ dimensional matrix:
$$
    shape(\mathbf{R}) = (T, n_l, d_1, d_2, \cdots)
$$
where $d$ is the dimension of the original input.
For compact visualisation, the results are aggregated, mean averaged in the first dimension of $\mathbf{R}$ which highlights the main parts of the input, that took role in increasing the activation of the given neuron.


\subsection{Results of visualization}

A pre-trained network holds generalized information in the weights and biases, and the approach mentioned in \emph{subsection \ref{method}}, produce great amount of possible representations of the networks' inner state. In this subsection MNIST samples enhanced by parameters derived from the \emph{MLP}s trained in \emph{subsection \ref{train}} are analyzed and compared - \textit{absolutely} non exhaustively, following intuitions like:

\begin{itemize}
    \item Where, and how does abstraction occur, from combining previous layers' activation patterns
    \item How does differently distributed nodes affect the above.
    \item Do fully connected networks converge to recognizing \emph{localized} patterns, like convolutional networks do
\end{itemize}

\subsection{Challenges} Because the parameter space is too wide, the main obstacle in visualization is to find out what to look for. For example visualizing the current best network that has the following dimensions: 
\begin{itemize}
    \item for each node in the network there is a $T \times 28 \times 28$ sized input set
    \item for each layer there is a given set of nodes $N = [784, 784, 100, 10]$ respectively
\end{itemize}

Totaling in $T\times(N_l)$ picture per layer, which are hard to deal with, \emph{(see Figure \ref{fig:dense})}. 
Especially when the neuron's favored patterns can be matched to different digits, 
the $T$ should be increased in order to get a more general picture when aggregating the amplified pictures.
On the other hand, the gradient ascent optimization cannot terminate on \emph{convex functions}, which is actually true for the activation of neural node's activation w.r.t to its input - meaning that the original image can be amplified as many times as possible, the output of the corresponding neuron $y_l^i$ will always increase with the number of iterations $I$. An example for applying the same method, but with 5 times the original number of iterations depicted on \emph{Figure \ref{fig:dense}} is shown on \emph{Figure \ref{fig:dense-100}}.
Therefore finding the best $I$ is also essential for extracting important information from the network.


\begin{figure}
    \centering
    \includegraphics[width=0.6\textwidth]{mean-noGA.png}
    \caption{The naive way to extract interpretable patterns is to gather those inputs which excites the most the given layer, and \textbf{mean average} it. The case depicted on the figure, is that the ideal input is not trivial as the listed candidates shows, especially in the first layers. By aggregating}
    \label{fig:mean}
\end{figure}


\begin{figure}
    \centering
    \includegraphics[width=0.9\textwidth]{200-dense-GA20.png}
    \caption{Iteratively enhanced pictures of the first hidden layer of a \textbf{[300-300-300]} network (first 200 are shown in \emph{row-major order}). The main question is, which of them holds important features, or needs further analysis.}
    \label{fig:dense}
\end{figure}


\begin{figure}
    \centering
    \includegraphics[width=0.9\textwidth]{200-dense-GA100.png}
    \caption{After increasing $T$, compared to fig. \ref{fig:dense} some of the amplified pictures gets noisy, which is not a problem in the first case. It is a trend in the early layers, to only focus on small, localized type of features, that yields different candidates, which makes the mean of the outcome noisy overall. Especially this is one important feature of \emph{MLPs} that initialized with random weights they converge towards recognizing localized features.}
    \label{fig:dense-100}
\end{figure}


\emph{\textbf{Note}: Deviating, or averaging the input samples is an important step to strengthen the features which the given node truly recognize, and drop out those which it does not (see Figure \ref{fig:mean}). For the rest of the case studies mean averaging is used for aggregation}

\textbf{Remainder}: For the following examples, a small subset of samples are shown, which were selected for the most revealing clues. Still, they may not represent all features of the network.


\subsection{Analyzing the best performing network}
The best result belongs to a network with hidden layer width: 784-784-100. 

The layers were visualized with the following parameters, the training set was used for choosing candidates:
\begin{itemize}
    \item[] $T=100$
    \item[] $\eta=0.1$
    \item[] $I=10, 50$
\end{itemize}