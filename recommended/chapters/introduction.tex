\chapter{Introduction}
\section{Overview}


\epigraph{\textit{No one knows what the right algorithm is, but it gives us hope that if we can discover some crude approximation of whatever this algorithm is and implement it on a computer, that can help us make a lot of progress.}}{\rightline{{\rm --- Andrew Ng}}}

Every time we are interacting with our environment, we get closer to understand it %still the closer we are the less we understand. 
However, as a byproduct, questions arise for which pure logical (mathematical) solution is not available, or the problem is larger than to be solved by current ones.
Luckily we can always turn back to nature for inspiration: biological systems have proven their efficiency, therefore their functions worths to be further analyzed even if there could be theoretically a better way to tackle obstacles. 
One of the many branches of artificial intelligence, Neural Network is based on the nerve systems of living organisms which is capable of self-learning.

Such a simple paradigm is playing a fundamental role in boosting the industry and researches of today, because introducing Machine Learning to any field of life results in great leap forward. 
Thanks to the recent technological advancements, even with the current computational capacity of a personal computer one can get encouraging results by simply exploring the core principles of the topic.

\textbf{Main motivation.} Using outer libraries \cite{TF, torch, caffe}, without diving deep into mathematical proofs, treating neural networks as black-boxes thousands of useful applications \cite{haykin2004comprehensive} are made. 
On the other hand, building such architectures from the very basics helps to clarify how simple units might be organized, taught \cite{werbos1994roots}, and function \cite{hornik1989multilayer} as a large system. 
As complexity arises, the processes in the network becomes unclear and brings (yields?) the question: what is the purpose of each node? Methods to visualize how activation patterns formulate, and what information is held within them has been already investigated \cite{yosinski2015understanding}, and applied to improve performance \cite{zeiler2014visualizing}. My studies are mainly rely on recent publications, and researches in field of computer vision. The design of my own Deep Learning framework library is influenced by hot off press tutorials \cite{Goodfellow-et-al-2016-Book, deeplearningdotnet, nnsdl, stanfordlectures, gibiansky} and open-sources \cite{TF, torch, caffe} available for anyone. 

\textbf{Primary goal.} With my work I intend to bring closer, and demystify cutting-edge concepts of applied neural networkings for larger audiences. 
On related works and case studies I want to show what can be done by simply starting from the drawing board

\section{Importance of understanding}
\section{Thesis outline and contributions}






%
%\section*{Biological Motivation}
%\section*{Real Life Applications}
%\section*{Black Box representation}
%\section{The Perceptron model}
%\section{Importance of understanding}
%\section{Visualization}
