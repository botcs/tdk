\chapter{Introduction}
\section{Overview}


\epigraph{\textit{No one knows what the right algorithm is, but it gives us hope that if we can discover some crude approximation of whatever this algorithm is and implement it on a computer, that can help us make a lot of progress.}}{\rightline{{\rm --- Andrew Ng}}}

Every time we are interacting with our environment, we get closer to understanding it. %still the closer we are the less we understand. 
However, as a byproduct, further questions arise, for which pure logical (or mathematical) solution is not available, or the problem is larger what we can solve.
Luckily we can always turn back to nature for inspiration: biological systems have proven their efficiency, therefore their functions are worth to be further analyzed -- even if there could be theoretically a better way to tackle obstacles. 
One of the many branches of artificial intelligence, Neural Networks are based on the nervous systems of living organisms which is capable of self-learning.

Such a simple paradigm is playing a fundamental role in boosting 
industry and research of today, because introducing Machine Learning to many fields of life seems to results in great leap forward. 
Thanks to the recent technological advancements, even with the current computational capacity of a personal computer one can get encouraging results by simply exploring the core principles of the topic. 

In this work I provide a detailed explanation of each step of building a library,
which is capable of reproducing current results of scientific researches.
Also present applications which were made with my own library and case studies in which I have achieved unique results.
Finally I propose the main innovation of my recent work: a simple algorithm for visualizing the \textbf{perceptive field} of each layer in a feed-forward network.

\paragraph{Primary goal.} With my work I intend to bring closer, and demystify cutting-edge concepts of applied neural networkings for larger audiences. 
On related works and case studies I want to show what can be done by simply going back to the drawing board

\clearpage
\section{Related Works and Literature}

\paragraph{Main motivation.} By using external libraries \cite{TF, torch, caffe}, without diving deep into mathematical proofs, and treating neural networks as black-boxes, thousands of useful applications \cite{haykin2004comprehensive} are made. 
On the other hand, building such architectures from the very basics helps to clarify how simple units might be organized~\cite{milo2002network}, taught \cite{werbos1994roots}, and function \cite{hornik1989multilayer} as a large system. 
As complexity arises, the processes in the network becomes unclear and brings the question: what is the purpose of each node? 
Methods to visualize how activation patterns formulate, and what information is held within them has been already investigated \cite{yosinski2015understanding}, and applied to improve performance \cite{zeiler2014visualizing}. 
My studies mainly rely on recent publications, and research in field of computer vision. The design of my own Deep Learning framework library is influenced by off-the-press tutorials \cite{Goodfellow-et-al-2016-Book, deeplearningdotnet, nnsdl, stanfordlectures, gibiansky} and open-source libraries \cite{TF, torch, caffe} available to anyone.

Considering course slides \cite{stanfordlectures, oxfordlectures} and textbooks \cite{Goodfellow-et-al-2016-Book, werbos1994roots, bengio2009learning} of established universities, popular websites \cite{deeplearningdotnet, pedregosa2011scikit}, 
articles \cite{lecun2015deep}, blogs \cite{gibiansky, karpathyblog} and vlogs \cite{vlog1} on Artificial Neural Networks one might find it hard to find a good point to start. 
Some content may offer formal description of the general machine learning problem, others try to clarify through analogies with the biological nerve system.
Tutorials, which I have found interesting, and the most helpful, had several common features which are important to adapt, when designing a neural network library.
Generally these common attributes are the following: 
they intend to be \emph{simple} as possible, 
have many \emph{intuitive} examples and analogies,
their \emph{interpretations} are not restricted to either universal approximators, 
or nervous systems of living organisms, but combines both aspect.

I have written the first implementation based on chapters of \cite{nnsdl}. 
What I have understood is the basic forward- and back-propagation \cite{werbos1994roots} concept of neural networks.
Since I was struggling to understand the coding tricks that the author involved 
I could not extend the source to any problem.
I decided to work out the mathematical formulation, to re-implement feed-forward networks from the beginning.
Reading a practical guide~\cite{karpathyblog} pointed out how can we represent \emph{gradient flow}, \emph{computational graphs} in a simple way.
His style of explaining complex theoretical concepts through very clear examples motivated me to
write my own guide which covers those parts that I was missing from the literature mentioned above.

When the library was working, the next step was writing demo applications, in which I have experimented with hyper-parameters.
These applications were based on \emph{Browser demos} of \cite{convnetjs}.
In this work only two of them is present, which are related to visualization.
Later I was experimenting with different methods of revealing the \emph{perceptive field}.
In \cite{breakingclass} I learned about how modern networks can be fooled with \emph{adversarial samples} which referred to \cite{goodfellow2014explaining} and \cite{nguyen2015deep}. 
Further investigation in the area revealed basic methods on understanding how each neuron works in a network \cite{yosinski2015understanding}.
I was inspired to develop my own algorithm while I was studying the novel architecture of DeconvNets \cite{zeiler2014visualizing}.

\clearpage
\section{Thesis outline and contributions}
In this work I summarize my recent studies on Neural Networks and Computer 
Vision. 
I present my current achievements in the field, the visualization 
techniques I have derived and developed from the latest 
self-educational sources and novel research.
I also provide a guide on how to build a library \textbf{from scratch} 
which I used for producing the present results.


In Chapter \textbf{\nameref{cp:design}} I explain the overall structure of the library and the considerations which led to them, and provide a detailed explanation on how I 
derived the main concepts of neural networks, from the very beginning supported with mathematical background. 
First I zoom in from a larger picture of working AI to exact challenges of 
Machine Learning: I disassemble neural networks in general, to basic 
building blocks, explain them through several examples and provide a 
formal mathematical definition which will be used in the following 
sections.
I put emphasis on linking the formal definitions to natural meanings of 
mathematical formulas (throughout rather informal explanations).
By providing intuitive theoretical considerations, from the perceptron 
model, I build up step-by-step the entire idea of \emph{Forward- \& 
Back-propagation} of Deep Neural Networks.
In the end, exact layers are described involving examples of practical applications.

In \textbf{\nameref{cp:results}}, I describe my research on 
visualization techniques via elementary case studies of \emph{Voice 
Recognition} and \emph{Image Classification}.
In the first half I summarize the main concepts and capabilities of single 
layered neural networks.
I start with the basic problems of voice recognition, namely the 
overlapping frequency domains, the irrelevant signals (noise) by training 
networks on pure sinusoid input samples.
I provide examples for the problem of overfitting and show how regularization techniques enhances the network performance.
After clarification, I explain the challenges of the voice recognition topic involving real recordings,
a case study that can be carried out at home by anyone, with the knowledge 
provided by my work.
In the second half, I show how feed-forward networks can be better understood via visualization with a simple algorithm, namely the \emph{Gradient Ascent}, which I have developed based on my previous studies of novel articles.
I have also explored the hyper-parameter space of fully connected architectures,
while training networks on the popular MNIST dataset.
I present my experiments with different shape of layers and methods for training, and share my interpretations of the results.
In the end an exhaustive investigation of different types of Gradient Ascent can be found involving research of revealing perceptive field of neuron each neuron in a given layer and generating adversarial samples from any kind of input which can be used for harnessing the network.

Finally, in the last chapters I summarize the research I have done so far and discuss the path forward.





%
%\section*{Biological Motivation}
%\section*{Real Life Applications}
%\section*{Black Box representation}
%\section{The Perceptron model}
%\section{Importance of understanding}
%\section{Visualization}
